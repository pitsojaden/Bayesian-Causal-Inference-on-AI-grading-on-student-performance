% Conclusion Section for: Bayesian Causal Inference on AI Grading Effects on Student Performance

\section{Conclusion}

This study provides causal evidence that AI grading intensity positively affects student assessment scores in online learning. Using hierarchical Bayesian mediation analysis on data from 25,471 students across 22 modules, we find a robust effect of approximately 45 points per standard deviation increase in AI intensity. Contrary to algorithm aversion predictions, this effect does not operate through reduced engagement. Only 4.4\% of the total effect is mediated by student behavior. The remaining 96\% reflects direct mechanisms, likely related to grading consistency or algorithmic characteristics.

These findings have practical implications for educational technology deployment. Institutions need not avoid AI grading based on engagement concerns. However, the possibility that higher scores reflect algorithmic leniency rather than genuine learning warrants continued attention. The disproportionate benefit for low-engagement students suggests that targeted deployment may maximize educational equity.

Methodologically, this study demonstrates the feasibility of Bayesian causal mediation analysis in educational settings where randomization is impractical. The analytical framework, including hierarchical random effects and formal sensitivity analyses, can be adapted to other questions at the intersection of technology and learning.
