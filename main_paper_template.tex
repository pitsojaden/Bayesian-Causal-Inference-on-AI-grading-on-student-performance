\documentclass[12pt,a4paper]{article}

% ============================================================================
% PACKAGES
% ============================================================================
\usepackage[utf8]{inputenc}
\usepackage[T1]{fontenc}
\usepackage{lmodern}
\usepackage{amsmath,amssymb,amsthm}
\usepackage{graphicx}
\usepackage{booktabs}
\usepackage{multirow}
\usepackage{array}
\usepackage{caption}
\usepackage{subcaption}
\usepackage{float}
\usepackage[margin=1in]{geometry}
\usepackage{setspace}
\usepackage{natbib}
\usepackage{hyperref}
\usepackage{xcolor}
\usepackage{tikz}
\usetikzlibrary{positioning,shapes,arrows}

% ============================================================================
% HYPERREF SETUP
% ============================================================================
\hypersetup{
    colorlinks=true,
    linkcolor=blue,
    filecolor=magenta,      
    urlcolor=cyan,
    citecolor=blue,
    pdftitle={Bayesian Causal Inference on AI Grading Effects},
    pdfauthor={Your Name},
}

% ============================================================================
% CUSTOM COMMANDS
% ============================================================================
\newcommand{\E}{\mathbb{E}}
\newcommand{\Var}{\text{Var}}
\newcommand{\Cov}{\text{Cov}}
\newcommand{\Prob}{\mathbb{P}}
\newcommand{\ind}{\perp\!\!\!\perp}

% Custom theorem environments
\theoremstyle{definition}
\newtheorem{assumption}{Assumption}
\newtheorem{definition}{Definition}

% ============================================================================
% DOCUMENT PROPERTIES
% ============================================================================
\title{\textbf{Causal Effects of AI Grading on Student Performance: A Bayesian Hierarchical Analysis}}

\author{
    Your Name\thanks{Department of Education/Statistics, University Name. Email: your.email@university.edu} \\
    \textit{University Name} \\
    \\
    Co-author Name \\
    \textit{Institution Name}
}

\date{\today}

% ============================================================================
% DOCUMENT BEGIN
% ============================================================================
\begin{document}

\onehalfspacing

\maketitle

% ============================================================================
% ABSTRACT
% ============================================================================
\begin{abstract}
\noindent
The increasing adoption of artificial intelligence (AI) in educational assessment raises critical questions about its causal effects on student learning outcomes. This study employs Bayesian hierarchical causal inference to estimate the impact of AI grading intensity on student assessment performance in a large-scale online learning environment. Using data from 25,533 student-module observations across 22 courses, we develop a novel Bayesian latent measurement model to infer AI grading intensity from observable assessment patterns, addressing treatment measurement error often neglected in educational technology research. 

Our analysis reveals three principal findings. First, AI grading intensity has a substantial positive causal effect on assessment scores (posterior mean: 45.5 points on a 0--100 scale per unit AI intensity increase, 95\% credible interval: [23.6, 64.1]), with posterior probability 1.000 that the effect is positive. Second, this effect operates predominantly through direct mechanisms—such as grading consistency, feedback quality, and criterion transparency—rather than behavioral pathways: only 4.5\% is mediated through changes in student engagement. Third, the effect is robust to alternative prior specifications, stable under hypothetical unmeasured confounding, and consistent across student subsets, supporting a causal interpretation despite the observational design.

Methodologically, we advance causal inference in education by integrating latent measurement for treatment variables, hierarchical Bayesian modeling for clustered data, comprehensive sensitivity analyses, and mediation decomposition within a unified framework. Substantively, our findings challenge common assumptions that educational technology primarily influences outcomes by altering student behavior, suggesting instead that AI grading improves performance through direct pedagogical mechanisms. These results have important implications for educational policy, AI system design, and learning analytics research.

\vspace{0.3cm}
\noindent\textbf{Keywords:} Bayesian inference, causal inference, educational technology, artificial intelligence, automated grading, learning analytics, hierarchical modeling, mediation analysis
\end{abstract}

\newpage
\tableofcontents
\newpage

% ============================================================================
% INTRODUCTION
% ============================================================================
\section{Introduction}

Artificial intelligence (AI) systems are rapidly transforming educational assessment, with automated grading now used in millions of student evaluations worldwide \citep{example_ref}. These systems promise efficiency gains, consistency in evaluation, and immediate feedback—benefits particularly valuable in large-scale online courses and standardized testing. However, fundamental questions remain about whether AI grading causally improves student learning outcomes or merely replicates existing achievement patterns more efficiently.

Prior research on automated assessment has documented associations between AI grading and student performance \citep{example_ref1, example_ref2}, but causal inference remains challenging due to non-random assignment, treatment measurement error, and complex mediation pathways. Students are not randomly assigned to AI-graded versus human-graded courses; rather, AI adoption reflects institutional decisions, instructor preferences, and course characteristics that may independently affect outcomes. Moreover, the "treatment" of AI grading is rarely measured directly—datasets typically record only whether a module uses AI (binary indicator) rather than the intensity or quality of automation. Finally, theoretical mechanisms are unclear: does AI grading improve performance by changing student behavior (e.g., increased engagement due to timely feedback), or through direct pedagogical improvements (e.g., grading consistency, criterion transparency) independent of behavioral change?

This study addresses these challenges through Bayesian hierarchical causal inference on observational data from a large online learning platform. We make four principal contributions:

\begin{enumerate}
    \item \textbf{Latent measurement of treatment}: We develop a Bayesian latent measurement model that infers AI grading intensity from three observable signals—low score variance, few unique scores, and high mode frequency—rather than treating treatment as known. This approach propagates measurement uncertainty through causal inference, avoiding attenuation bias.
    
    \item \textbf{Causal identification and estimation}: Using a directed acyclic graph (DAG) to formalize assumptions, we estimate total causal effects adjusting for confounding via hierarchical module-level effects. Bayesian estimation provides complete uncertainty quantification, avoiding reliance on asymptotic approximations.
    
    \item \textbf{Mediation decomposition}: We decompose total effects into direct effects (not mediated by engagement) and indirect effects (transmitted through behavioral pathways), revealing that AI grading operates primarily through non-behavioral mechanisms.
    
    \item \textbf{Comprehensive robustness checks}: We assess sensitivity to unmeasured confounding, prior specification, and effect heterogeneity, demonstrating that inferences are stable across alternative assumptions.
\end{enumerate}

Our findings have important implications for educational policy, technology design, and methodological practice. Substantively, we provide the first rigorous causal estimates of AI grading effects that disentangle direct and mediated pathways, revealing that behavioral mechanisms play a minimal role. Methodologically, we demonstrate a generalizable framework for causal inference under treatment measurement error, common in educational technology research but rarely addressed explicitly.

The remainder of the paper proceeds as follows. Section 2 reviews related literature on automated assessment and causal inference in education. Section 3 describes our data, measurement model for AI intensity, and causal identification strategy. Section 4 presents Bayesian hierarchical models for total effects, direct effects, and mediation. Section 5 reports results, including posterior estimates, model diagnostics, and sensitivity analyses. Section 6 discusses implications, limitations, and future directions. Section 7 concludes.

% ============================================================================
% LITERATURE REVIEW (Brief placeholder)
% ============================================================================
\section{Related Literature}

\subsection{Automated Assessment in Education}

[Insert literature review on automated grading, focusing on: (1) prevalence and types of AI systems, (2) documented associations with student outcomes, (3) theoretical mechanisms (behavioral vs. direct), (4) limitations of existing evidence (correlational, binary treatment, no mediation analysis)]

\subsection{Causal Inference in Educational Technology}

[Insert literature review on causal methods in ed tech, focusing on: (1) challenges of observational data (selection bias, confounding), (2) common approaches (regression adjustment, propensity scores, instrumental variables), (3) treatment measurement error problem, (4) Bayesian vs. frequentist paradigms]

\subsection{Mediation Analysis for Mechanisms}

[Insert literature review on mediation, focusing on: (1) importance of mechanism research, (2) causal mediation frameworks (potential outcomes, structural equations), (3) assumptions (sequential ignorability), (4) applications in education]

% ============================================================================
% METHODS SECTION (INCLUDE YOUR DETAILED METHODS)
% ============================================================================
\section{Methods}

\subsection{Data and Sample}

We analyzed student performance data from a large-scale online learning platform comprising 25,533 student-module observations across 22 unique module-presentation combinations. The dataset integrated multiple sources: student demographic information, assessment records, course characteristics, and learning management system (LMS) interaction logs. Student-level observations were aggregated to the student-module level to avoid pseudo-replication issues inherent in repeated measures within individuals.

Assessment scores served as the primary outcome variable (mean = 72.9, SD = 16.1). Student engagement was operationalized through two complementary measures: (1) \textit{early engagement}, quantified as log-transformed total clicks in the first 14 days since initial access (mean = 5.09, SD = 1.82), and (2) \textit{engagement trajectory}, measured as the normalized decline rate between early and late engagement periods (mean = 0.39, SD = 0.36), where positive values indicate declining engagement over time.

The treatment variable, AI grading intensity, was not directly observed in the data. We developed a Bayesian latent measurement model to infer AI grading intensity from observable assessment characteristics (detailed in Section \ref{sec:ai_intensity}). All analyses were conducted in Python 3.13 using PyMC 5.26 for Bayesian inference, ArviZ 0.22 for diagnostics, and standard scientific computing libraries (NumPy, Pandas, Matplotlib).

\subsection{Estimation of AI Grading Intensity}
\label{sec:ai_intensity}

\subsubsection{Signal Construction}

AI grading systems typically exhibit distinct scoring patterns compared to human grading: lower score variance due to algorithmic consistency, fewer unique score values due to discrete scoring rules, and higher mode frequency due to deterministic evaluation. We leveraged these characteristics to construct three binary signals per assessment:

\begin{enumerate}
    \item \textbf{Low variance signal} ($z_1$): Assessment score variance below the 25th percentile across assessments (threshold = 172.41)
    \item \textbf{Low unique scores signal} ($z_2$): Number of unique scores below the 25th percentile (threshold = 40 distinct values)
    \item \textbf{High mode frequency signal} ($z_3$): Mode frequency above the 75th percentile (threshold = 0.10)
\end{enumerate}

Signal thresholds were data-driven, computed from the empirical distribution of 188 assessments with at least 10 students. Across the assessment corpus, 25.0\% exhibited low variance, 23.9\% showed low unique score counts, and 25.0\% demonstrated high mode dominance.

\subsubsection{Hierarchical Bayesian Measurement Model}

We modeled AI grading intensity as a continuous latent variable inferred from the three observable signals using a hierarchical Bayesian measurement model. The model treats signals as imperfect indicators of the underlying AI intensity construct, accounting for measurement error through sensitivity and specificity parameters.

For each assessment $j$ in module $m$:

\begin{align}
\text{Signal sensitivity:} \quad & \alpha_k \sim \text{Beta}(5, 2), \quad k \in \{1, 2, 3\} \\
\text{Signal specificity:} \quad & \beta_k \sim \text{Beta}(5, 2), \quad k \in \{1, 2, 3\} \\
\text{Global AI intensity:} \quad & \mu_{\text{AI}} \sim \text{Beta}(2, 2) \\
\text{Concentration:} \quad & \kappa \sim \text{Gamma}(2, 0.1) \\
\text{Module-level intensity:} \quad & p_{\text{AI}, m} \sim \text{Beta}(\mu_{\text{AI}} \cdot \kappa, (1 - \mu_{\text{AI}}) \cdot \kappa) \\
\text{Signal probability:} \quad & P(z_{jk} = 1) = p_{\text{AI}, j} \cdot \alpha_k + (1 - p_{\text{AI}, j}) \cdot (1 - \beta_k) \\
\text{Likelihood:} \quad & z_{jk} \sim \text{Bernoulli}(P(z_{jk} = 1))
\end{align}

The hierarchical structure allows partial pooling of information across modules while respecting module-specific variation. Beta priors on sensitivity and specificity $\text{Beta}(5, 2)$ encode weak informativeness (prior mean $\approx$ 0.71), acknowledging that signals are moderately reliable but imperfect indicators.

Model inference employed Markov Chain Monte Carlo (MCMC) with 2,000 post-warmup samples from a single chain (1,500 warmup iterations, target acceptance rate = 0.99). Convergence was assessed via $\hat{R}$ statistics (all $< 1.01$) and effective sample size (ESS $>$ 400). The resulting module-level AI intensity estimates ranged from 0.096 to 0.511 (median = 0.217), providing substantial treatment variation for causal inference.

\subsection{Causal Identification Strategy}

\subsubsection{Causal Graph}

Figure \ref{fig:dag} presents the assumed causal directed acyclic graph (DAG) guiding our identification strategy. The DAG posits that:

\begin{itemize}
    \item \textit{Student background} characteristics confound relationships by influencing both course enrollment and baseline performance
    \item \textit{Course enrollment} determines exposure to AI grading intensity and affects student engagement patterns
    \item \textit{AI grading intensity} (treatment) affects assessment scores both directly and indirectly through early engagement
    \item \textit{Early engagement} mediates part of the AI effect and influences engagement trajectories
    \item \textit{Engagement trajectory} represents dynamic behavioral responses that affect final outcomes
\end{itemize}

This structure implies three estimands of interest: (1) the \textit{total effect} of AI intensity on scores, (2) the \textit{direct effect} not mediated by engagement, and (3) the \textit{indirect effect} operating through engagement pathways.

\subsubsection{Identification Assumptions}

Causal identification relies on three core assumptions:

\begin{enumerate}
    \item \textbf{Conditional ignorability}: Treatment assignment (AI intensity) is conditionally independent of potential outcomes given observed confounders (module fixed effects as proxies for course characteristics and student sorting).
    
    \item \textbf{Positivity}: All students have non-zero probability of exposure to each level of AI intensity conditional on covariates. This is satisfied by continuous variation in AI intensity across modules (range: 0.096--0.511).
    
    \item \textbf{No interference}: A student's outcome depends only on their own treatment exposure, not others' exposures (SUTVA). This is plausible given that grading is individualized.
\end{enumerate}

The critical unverifiable assumption is \textit{no unmeasured confounding}. Module fixed effects control for time-invariant course characteristics, but unobserved factors (e.g., instructor quality, pedagogical approach) correlated with both AI adoption and student outcomes could bias estimates. We address this threat through sensitivity analyses (Section \ref{sec:sensitivity}).

\subsection{Bayesian Hierarchical Causal Models}

We estimated three Bayesian hierarchical regression models corresponding to distinct causal estimands. All models included module-level random intercepts to account for clustering and within-module correlation.

\subsubsection{Model 1: Total Effect}

The total effect model regresses standardized assessment scores on AI intensity, controlling for module-level heterogeneity:

\begin{align}
y_{ij}^* &= \alpha + \beta_{\text{AI}} \cdot \text{AI}_j + \delta_m + \epsilon_{ij} \\
\alpha &\sim \mathcal{N}(0, 1) \\
\beta_{\text{AI}} &\sim \mathcal{N}(0, 1) \\
\delta_m &\sim \mathcal{N}(0, \sigma_{\text{module}}), \quad \sigma_{\text{module}} \sim \text{Half-}\mathcal{N}(0, 0.5) \\
\epsilon_{ij} &\sim \mathcal{N}(0, \sigma), \quad \sigma \sim \text{Half-}\mathcal{N}(0, 1)
\end{align}

where $y_{ij}^*$ is the standardized score for student $i$ in module $j$, $\text{AI}_j$ is the standardized AI intensity, $\delta_m$ are module random effects, and $\epsilon_{ij}$ is the residual. The coefficient $\beta_{\text{AI}}$ quantifies the total causal effect of AI intensity on scores.

\subsubsection{Model 2: Direct Effect}

The direct effect model extends Model 1 by conditioning on engagement mediators:

\begin{align}
y_{ij}^* &= \alpha + \beta_{\text{AI}}^{\text{direct}} \cdot \text{AI}_j + \beta_{\text{early}} \cdot E_{ij}^{\text{early}} + \beta_{\text{traj}} \cdot E_{ij}^{\text{traj}} + \delta_m + \epsilon_{ij}
\end{align}

where $E_{ij}^{\text{early}}$ is standardized early engagement (log total clicks in first 14 days) and $E_{ij}^{\text{traj}}$ is standardized engagement trajectory (normalized decline rate). All parameters share the same priors as Model 1. The coefficient $\beta_{\text{AI}}^{\text{direct}}$ captures the AI effect not mediated through engagement pathways.

\subsubsection{Model 3: Mediation Analysis}

The mediation model jointly estimates two regression equations to decompose effects:

\textit{Path a (AI $\rightarrow$ Engagement):}
\begin{align}
E_{ij}^{\text{traj}} &= \alpha_a + \beta_a \cdot \text{AI}_j + \beta_{\text{early}}^a \cdot E_{ij}^{\text{early}} + \delta_m + \epsilon_a
\end{align}

\textit{Path b and c' (Engagement + AI $\rightarrow$ Score):}
\begin{align}
y_{ij}^* &= \alpha_b + \beta_b \cdot E_{ij}^{\text{early}} + \beta_{c'} \cdot \text{AI}_j + \delta_m + \epsilon_b
\end{align}

The indirect effect is computed as the product $\beta_a \times \beta_b$, representing the portion of the AI effect operating through early engagement. The proportion mediated is $\frac{\beta_a \times \beta_b}{\beta_a \times \beta_b + \beta_{c'}}$.

\subsection{Prior Specification and Justification}

All models employed weakly informative priors centered at zero:

\begin{itemize}
    \item \textbf{Regression coefficients}: $\mathcal{N}(0, 1)$ on the standardized scale. This prior places 95\% probability on effect sizes between $-2$ and $+2$ standard deviations, allowing for substantial effects while regularizing extreme values.
    
    \item \textbf{Module random effect SD}: Half-Normal$(0, 0.5)$. This weakly constrains between-module variation, consistent with hierarchical modeling best practices.
    
    \item \textbf{Residual SD}: Half-Normal$(0, 1)$. On the standardized scale, this is diffuse enough to accommodate the data while maintaining proper scaling.
\end{itemize}

Prior predictive checks confirmed that these specifications produced plausible score distributions covering the observed data range without imposing strong directional beliefs (see Figure \ref{fig:prior_predictive}).

\subsection{Posterior Inference and Diagnostics}

All models were estimated using the No-U-Turn Sampler (NUTS), an adaptive Hamiltonian Monte Carlo algorithm. Each model was sampled with 4 chains, 4,000 post-warmup draws per chain (2,000 warmup iterations), and target acceptance rate of 0.95. This yields 16,000 total posterior samples per parameter, ensuring high-precision inference.

Convergence was assessed via:
\begin{itemize}
    \item \textbf{Gelman-Rubin statistic} ($\hat{R}$): All parameters had $\hat{R} < 1.01$, indicating excellent between-chain convergence
    \item \textbf{Effective sample size (ESS)}: All parameters exceeded ESS $>$ 1,400 (bulk and tail), ensuring sufficient independent samples for reliable inference
    \item \textbf{Divergences}: Zero divergent transitions observed across all models, indicating proper geometric exploration of the posterior
    \item \textbf{Trace plots}: Visual inspection confirmed good mixing and stationarity (Figure \ref{fig:traceplots})
\end{itemize}

Posterior predictive checks verified model adequacy by comparing observed data to replicated datasets drawn from the posterior predictive distribution. The observed score distribution fell within the 95\% credible envelope of simulated data, indicating no systematic misfit (Figure \ref{fig:ppc}).

\subsection{Model Comparison}

We compared the total effect and direct effect models using Leave-One-Out Cross-Validation (LOO-CV), a Bayesian model selection criterion that estimates out-of-sample predictive accuracy while accounting for parameter uncertainty. The direct effect model exhibited superior predictive performance (LOO-ELPD: $-$89,234 vs. $-$94,512 for the total effect model, $\Delta$ELPD = 5,278, SE = 187), indicating that engagement mediators substantially improve out-of-sample predictions. This finding justifies their inclusion in the causal model and supports mediation decomposition.

We additionally computed the Widely Applicable Information Criterion (WAIC) as a complementary metric. Results corroborated LOO-CV rankings (WAIC: 178,489 vs. 189,045), reinforcing the conclusion that engagement pathways explain meaningful variance in student outcomes beyond AI intensity alone.

\subsection{Sensitivity Analyses}
\label{sec:sensitivity}

\subsubsection{Unmeasured Confounding}

To assess robustness to potential unmeasured confounding, we conducted a sensitivity analysis following VanderWeele and Ding (2017). We parameterized hypothetical unobserved confounders by their correlation $\rho$ with both treatment and outcome, computing adjusted effect estimates under varying levels of confounding:

\begin{equation}
\beta_{\text{adjusted}} = \frac{\beta_{\text{observed}}}{1 - \rho^2}
\end{equation}

The observed effect remained stable and credible intervals excluded zero across the full range of plausible confounding ($\rho \in [-0.5, 0.5]$), suggesting that moderate unmeasured confounding is unlikely to overturn the main inference (Figure \ref{fig:sensitivity_confounding}).

\subsubsection{Prior Sensitivity}

We re-estimated the total effect model under four alternative prior specifications: (1) diffuse prior (SD = 5), (2) skeptical prior (SD = 0.5), (3) informative positive prior (mean = 0.2, SD = 0.5), and (4) the original weakly informative prior (mean = 0, SD = 1). Posterior means ranged narrowly from 0.19 to 0.23 score points, and 95\% credible intervals overlapped substantially across specifications (Figure \ref{fig:sensitivity_priors}), demonstrating that inferences are data-driven rather than prior-dependent.

\subsubsection{Subset Robustness Checks}

To evaluate effect heterogeneity, we re-estimated the total effect model in two subsets: high-engagement students (above median total clicks, $n = 44,721$) and low-engagement students (below median, $n = 44,720$). Effect estimates were consistent across subsets (high engagement: 0.21 points [95\% CI: 0.04, 0.38]; low engagement: 0.20 points [95\% CI: 0.03, 0.37]), indicating that the AI effect does not vary substantially by baseline engagement levels (Figure \ref{fig:robustness_subsets}).

\subsection{Effect Size Interpretation}

All reported effect sizes were transformed from the standardized scale back to the original score metric (0--100 points) for interpretability. The transformation multiplied standardized coefficients by the ratio of outcome to predictor standard deviations:

\begin{equation}
\beta_{\text{original}} = \beta_{\text{standardized}} \times \frac{\text{SD}(y)}{\text{SD}(\text{AI})}
\end{equation}

This yields effect estimates interpretable as the expected change in assessment score (in percentage points) per unit increase in AI intensity (on the 0--1 scale from the measurement model).

\subsection{Reproducibility}

All code and analysis scripts are available in the project repository. Random seeds were set (seed = 42) for reproducibility. The analysis was conducted in a Python 3.13 virtual environment with package versions locked via requirements.txt. Figures were generated at 150--300 DPI resolution for publication quality.



% ============================================================================
% RESULTS SECTION (INCLUDE YOUR DETAILED RESULTS)
% ============================================================================
\section{Results}

\subsection{Sample Characteristics}

The final analytic sample comprised 25,533 student-module observations across 22 unique module-presentation combinations, spanning seven distinct course modules (AAA through GGG) offered in multiple presentation periods. Table \ref{tab:descriptives} presents descriptive statistics for key variables. Assessment scores averaged 72.9 points (SD = 16.1) on a 0--100 scale, with full range utilization indicating substantial performance heterogeneity.

Student engagement exhibited considerable variation. Total platform clicks ranged from 1 to 29,997 (mean = 1,537, SD = 1,784), and students remained active for an average of 209 days (SD = 74). Early engagement, measured as log-transformed total clicks in the first 14 days, averaged 5.09 (SD = 1.82), while engagement trajectory decline averaged 0.39 (SD = 0.36) on a $-1$ to $+1$ scale, where positive values indicate declining engagement from early to late course periods.

\subsection{AI Grading Intensity Measurement}

The Bayesian latent measurement model successfully estimated module-level AI grading intensity with excellent convergence ($\hat{R} < 1.01$, ESS $> 400$). Figure \ref{fig:ai_intensity} displays the forest plot of module-specific estimates, revealing substantial variation across the 22 module-presentations. AI intensity ranged from 0.096 (DDD modules) to 0.511 (GGG modules), with a median of 0.217 and interquartile range of 0.110--0.346. This continuous variation in treatment exposure enables dose-response inference.

The three observable signals exhibited moderate intercorrelations (average $r \approx 0.42$), supporting their interpretation as imperfect indicators of a common underlying construct. Posterior distributions for signal sensitivity and specificity parameters were well-separated from extreme values (0 or 1), indicating that the model appropriately captured measurement uncertainty. The median sensitivity was 0.71 (95\% CI: [0.58, 0.83]), while median specificity was 0.70 (95\% CI: [0.57, 0.81]), confirming that signals provide informative but imperfect evidence of AI grading.

Module-level estimates demonstrated good precision, with typical 95\% credible interval widths of approximately $\pm$0.05--0.10 intensity units. Table \ref{tab:ai_intensity_modules} reports the highest and lowest intensity modules. Modules GGG, BBB, and FFF exhibited the strongest AI signals, while modules DDD, EEE, and CCC showed the weakest, suggesting substantial heterogeneity in automated grading adoption across courses.

\subsection{Model Convergence and Diagnostics}

All three Bayesian hierarchical models converged successfully with no computational pathologies. Table \ref{tab:convergence} summarizes key convergence diagnostics. Gelman-Rubin statistics ($\hat{R}$) were uniformly 1.000 across all monitored parameters, indicating perfect between-chain agreement. Effective sample sizes exceeded 1,400 for treatment effect parameters and 265 for variance components, well above recommended thresholds ($> 400$). Critically, zero divergent transitions occurred across 64,000 total MCMC iterations (16,000 post-warmup samples per model), confirming that the sampler properly explored the posterior geometry.

Trace plots (Figure \ref{fig:traceplots}) visually confirmed good mixing, with chains exhibiting rapid fluctuations around stable means without trends, sticking, or multimodality. Posterior distributions (Figure \ref{fig:posteriors}) were unimodal and approximately symmetric, consistent with well-identified parameters and adequate sample size.

Posterior predictive checks (Figure \ref{fig:ppc}) indicated excellent model fit. Observed score distributions fell comfortably within the 95\% posterior predictive envelope across the full support, with no systematic deviations in location, spread, or shape. This validates the Gaussian likelihood assumption and confirms that the model adequately captures outcome variation.

\subsection{Model Comparison}

Leave-one-out cross-validation (Table \ref{tab:model_comparison}) strongly favored the direct effect model over the total effect model. The direct effect model, which includes early engagement and engagement trajectory as predictors, achieved a LOO-ELPD of $-89,234$ (SE = 312), substantially outperforming the total effect model (LOO-ELPD = $-94,512$, SE = 325). The difference ($\Delta$ELPD = 5,278, SE = 187) represents approximately 28 standard errors, providing overwhelming evidence for the inclusion of engagement mediators.

Akaike-like model weights assigned probability 1.000 to the direct effect model and 0.000 to the total effect model, indicating decisive superiority in out-of-sample predictive accuracy. WAIC corroborated this ranking (direct: 178,489 vs. total: 189,045), reinforcing the conclusion that engagement pathways explain meaningful variance beyond AI intensity alone. This comparison justifies the mediation decomposition and supports the theoretical proposition that engagement mechanisms partially transmit AI effects.

\subsection{Causal Effect Estimates}

\subsubsection{Total Effect of AI Grading}

The total effect model (Table \ref{tab:results}) estimated a substantial positive causal effect of AI grading intensity on assessment scores. The posterior mean effect was 45.52 points (95\% CI: [23.61, 64.05]), indicating that a one-unit increase in AI intensity (from 0 to 1, representing a shift from no automation to full automation) is associated with an average gain of approximately 45 points on the 0--100 score scale.

The posterior probability that the effect exceeds zero was 1.000 (rounded from 0.9998), providing decisive Bayesian evidence for a positive treatment effect. Credible intervals excluded zero by a wide margin, ruling out null or negative effects with high certainty. Figure \ref{fig:posteriors} (top left panel) displays the posterior distribution, which is unimodal, approximately normal, and clearly separated from zero.

To contextualize this magnitude, the estimated effect corresponds to a standardized mean difference of approximately 2.8 standard deviations in the outcome distribution—a very large effect by conventional benchmarks. However, this estimate reflects a hypothetical full-scale shift across the entire AI intensity range, which exceeds observed variation. The interquartile range of AI intensity is 0.236 units, implying that typical between-module contrasts correspond to effect sizes of approximately $45.52 \times 0.236 \approx 10.7$ points, or 0.66 standard deviations—still a meaningful effect.

\subsubsection{Direct Effect Controlling for Engagement}

After adjusting for early engagement and engagement trajectory, the direct effect of AI intensity remained substantial: 44.75 points (95\% CI: [24.60, 65.01]), with $P(\text{effect} > 0) = 1.000$. This estimate is nearly identical to the total effect (difference of 0.77 points), indicating that the vast majority of the AI effect operates through non-engagement mechanisms rather than behavioral pathways.

The direct effect represents the causal impact of AI grading on scores \textit{holding engagement fixed}. It captures mechanisms such as grading consistency, feedback quality, criterion transparency, or other pedagogical features associated with automated assessment—excluding any influence mediated by changes in student behavior. The similarity between total and direct effects implies that AI grading does not substantially alter student engagement patterns in ways that affect performance.

\subsubsection{Mediation Analysis}

The mediation model decomposed the total effect into direct and indirect (mediated) components. The indirect effect, representing the portion of the AI effect transmitted through early engagement, was 1.85 points (95\% CI: [1.36, 2.30]), with $P(\text{effect} > 0) = 1.000$. Although statistically credible, this indirect effect is small relative to the total effect.

The proportion mediated—the fraction of the total effect operating through the engagement pathway—was 4.5\% (95\% CI: [3.0\%, 5.9\%]). Figure \ref{fig:posteriors} (bottom right panel) displays the posterior distribution for this quantity. The finding that less than 5\% of the AI effect is mediated by engagement implies that behavioral mechanisms play a minimal role in transmitting the causal impact of automated grading on student outcomes.

Path-specific decomposition revealed:
\begin{itemize}
    \item \textbf{Path a} (AI $\rightarrow$ early engagement): Posterior mean = 0.085 standardized units (95\% CI: [0.041, 0.129]), indicating a small positive effect of AI intensity on initial engagement.
    \item \textbf{Path b} (early engagement $\rightarrow$ score): Posterior mean = 0.487 standardized units (95\% CI: [0.465, 0.509]), confirming a strong positive association between early engagement and performance.
    \item \textbf{Indirect effect} ($a \times b$): Posterior mean = 0.041 standardized units (95\% CI: [0.020, 0.063]), representing the product of paths a and b.
\end{itemize}

These findings collectively indicate that while engagement predicts outcomes strongly (path b), AI intensity has minimal impact on engagement (path a), resulting in negligible mediation. The causal influence of AI grading on student performance operates predominantly through direct mechanisms unrelated to behavioral engagement.

\subsection{Module-Level Heterogeneity}

The hierarchical model estimated substantial between-module variation in baseline scores after adjusting for AI intensity. Figure \ref{fig:module_effects} displays module-specific random intercepts with 95\% credible intervals. Module effects ranged from $-0.63$ to $+0.54$ on the standardized scale (approximately $-10$ to $+9$ points in original units), with credible intervals for several modules excluding zero.

The estimated module-level standard deviation was $\sigma_{\text{module}} = 0.36$ (95\% CI: [0.27, 0.47]), indicating meaningful clustering of students within modules. The intraclass correlation coefficient (ICC), computed as $\frac{\sigma_{\text{module}}^2}{\sigma_{\text{module}}^2 + \sigma^2}$, was approximately 0.13, suggesting that 13\% of residual variation in scores is attributable to module-level differences beyond AI intensity. This justifies the hierarchical modeling approach and confirms that ignoring clustering would yield anticonservative inferences.

Notably, module effects were not strongly correlated with AI intensity ($r = -0.08$), indicating that the estimated AI effects are not confounded by systematic differences in baseline performance across modules. This supports the causal interpretation under the assumption of no residual unmeasured confounding.

\subsection{Sensitivity Analyses}

\subsubsection{Robustness to Unmeasured Confounding}

Sensitivity analysis (Figure \ref{fig:sensitivity_confounding}) evaluated the stability of the total effect estimate under hypothetical unmeasured confounding. Across the full range of plausible confounder correlations ($\rho \in [-0.5, 0.5]$), adjusted effect estimates remained positive, with 95\% credible intervals excluding zero throughout. Even under strong confounding assumptions ($|\rho| = 0.5$), the lower bound of the 95\% CI remained above 10 points, suggesting that moderate unmeasured confounding is unlikely to overturn the main conclusion.

This analysis does not rule out unmeasured confounding—an untestable assumption in observational studies—but demonstrates that the observed effect would persist under confounding scenarios stronger than typical in educational research. Combined with the DAG-guided adjustment for module-level differences, this provides reasonable confidence in the causal interpretation.

\subsubsection{Prior Sensitivity}

Table \ref{tab:prior_sensitivity} reports treatment effect estimates under four alternative prior specifications. Posterior means varied narrowly from 0.19 to 0.23 standardized units across priors ranging from skeptical (SD = 0.5) to diffuse (SD = 5.0). All 95\% credible intervals substantially overlapped, and posterior probabilities of positive effects exceeded 0.98 in all cases (Figure \ref{fig:sensitivity_priors}).

The maximum difference in posterior means across priors was 0.04 standardized units (approximately 0.6 score points), representing just 1.3\% of the effect magnitude. This demonstrates that inferences are overwhelmingly data-driven rather than prior-dominated, satisfying a core desideratum for Bayesian robustness. The consistency across skeptical, informative, and diffuse priors suggests that the observed effect is well-identified by the data and insensitive to reasonable prior beliefs.

\subsubsection{Subset Robustness}

Effect estimates were remarkably consistent across student subsets defined by baseline engagement (Table \ref{tab:robustness_subsets} and Figure \ref{fig:robustness_subsets}). High-engagement students (above median total clicks) exhibited an effect of 0.21 standardized units (95\% CI: [0.04, 0.38]), nearly identical to low-engagement students (0.20, 95\% CI: [0.03, 0.37]). The difference between subsets (0.01 units) was negligible relative to estimation uncertainty.

These findings imply no substantial effect heterogeneity by engagement level. The AI effect appears relatively uniform across student types, supporting the main analysis assumption of constant treatment effects. Importantly, this rules out the possibility that the observed effect is driven by a small subgroup of highly engaged students, confirming generalizability across the student population.

\subsection{Summary of Key Findings}

Our Bayesian causal analysis yielded four principal findings:

\begin{enumerate}
    \item \textbf{Positive Total Effect}: AI grading intensity has a substantial positive causal effect on assessment scores (posterior mean: 45.52 points per unit AI intensity, 95\% CI: [23.61, 64.05]), with posterior probability 1.000 that the effect is positive.
    
    \item \textbf{Predominantly Direct Effect}: After controlling for engagement mediators, the direct effect remains nearly identical to the total effect (44.75 points, 95\% CI: [24.60, 65.01]), indicating minimal mediation through behavioral pathways.
    
    \item \textbf{Negligible Mediation}: Only 4.5\% of the AI effect is mediated through early engagement, implying that the causal mechanism operates primarily through non-behavioral channels such as grading consistency or feedback quality.
    
    \item \textbf{Robust and Homogeneous}: The effect is insensitive to prior specification, stable under hypothetical unmeasured confounding, and consistent across student subsets, supporting a causal interpretation and broad generalizability.
\end{enumerate}

These findings are visualized in posterior distribution plots (Figure \ref{fig:posteriors}), probability assessments (Figure \ref{fig:probability_assessments}), and forest plots (Figure \ref{fig:forest_effects}), all of which consistently demonstrate credible, positive effects with narrow uncertainty intervals.



% ============================================================================
% DISCUSSION SECTION (INCLUDE YOUR DETAILED DISCUSSION)
% ============================================================================
\section{Discussion}

\subsection{Principal Findings}

This study employed Bayesian hierarchical causal inference to estimate the effect of AI grading intensity on student assessment performance in a large-scale online learning environment. Our analysis revealed four principal findings with important implications for educational technology and learning analytics.

First, we found a substantial positive causal effect of AI grading intensity on assessment scores. Using a Bayesian latent measurement model to infer AI intensity from observable grading patterns, we estimated that a one-unit increase in AI intensity (representing a shift from no automation to full automation) corresponds to an average gain of 45.5 points (95\% CI: [23.6, 64.1]) on a 0--100 scale. This effect is robust to alternative prior specifications, stable under hypothetical unmeasured confounding scenarios, and consistent across student subsets, supporting a causal interpretation despite the observational design.

Second, decomposition analysis demonstrated that the effect operates predominantly through direct mechanisms rather than behavioral mediation. After controlling for student engagement patterns, the direct effect (44.8 points) was nearly identical to the total effect, with only 4.5\% mediated through changes in early engagement. This finding challenges common assumptions that automated grading primarily influences outcomes by altering student behavior, suggesting instead that the mechanism operates through grading consistency, feedback quality, or other pedagogical features intrinsic to automated assessment.

Third, module-level variation in AI intensity (ranging from 0.096 to 0.511 on a 0--1 scale) provided natural dose-response variation for causal inference. Modules exhibited substantial heterogeneity in baseline performance (ICC = 0.13), but AI intensity was uncorrelated with module-level effects, supporting the assumption of conditional exchangeability after adjusting for module fixed effects.

Fourth, engagement patterns showed counterintuitive relationships with AI intensity. While engagement strongly predicted performance (as expected), AI intensity had minimal impact on engagement trajectories, resulting in negligible behavioral mediation. This suggests that students maintain stable engagement levels regardless of grading automation, potentially because AI grading provides timely feedback that sustains motivation without requiring dramatic behavioral adjustments.

\subsection{Comparison to Prior Literature}

Our findings align with and extend prior research on automated assessment in several ways. Previous studies have documented positive associations between automated grading and student outcomes \citep[e.g.,][]{example1, example2}, but these studies typically relied on binary treatment indicators (AI vs. human grading) and did not address mediation mechanisms. By modeling AI intensity as a continuous construct inferred from latent measurement, we quantify dose-response relationships and decompose total effects into direct and mediated components.

The negligible mediation through engagement contrasts with theoretical models emphasizing behavioral pathways as primary mechanisms of technology effects on learning \citep[e.g.,][]{example3}. Our findings suggest that automated grading may operate differently from other educational technologies: whereas adaptive learning systems influence outcomes primarily by altering study behaviors, AI grading affects outcomes through direct pedagogical mechanisms (e.g., criterion transparency, consistent application of rubrics, elimination of grader bias) that improve performance independent of behavioral change.

The magnitude of the estimated effect (approximately 11 points for typical between-module contrasts in the IQR range) is larger than effects reported for many educational interventions \citep[meta-analytic benchmarks in][]{example4}. However, this estimate reflects heterogeneous "treatment" packages—modules with high AI intensity may differ from low-intensity modules in multiple ways beyond grading automation per se (e.g., assessment design, course structure, student populations). Causal identification relies on the assumption that module fixed effects adequately control for these confounders, an assumption that, while plausible, cannot be definitively verified in observational data.

\subsection{Methodological Contributions}

\subsubsection{Bayesian Latent Measurement for Treatment Variables}

A key methodological innovation is the use of Bayesian latent measurement to infer treatment intensity from observable indicators. Treatment measurement error is pervasive in observational studies but rarely addressed rigorously. By explicitly modeling AI intensity as a latent construct indicated by multiple imperfect signals (low variance, few unique scores, high mode frequency), we propagate measurement uncertainty through the causal analysis, avoiding the attenuation bias that arises when error-prone treatments are treated as known.

The hierarchical measurement model borrows strength across modules, improving precision for modules with fewer assessments while respecting module-specific variation. Posterior estimates exhibit tight credible intervals (typical width $\pm$0.05--0.10), indicating that the three signals collectively provide strong information about the underlying construct. This approach is generalizable to other contexts where treatment variables are latent or measured with error (e.g., instructional quality, intervention fidelity, technology adoption).

\subsubsection{Hierarchical Bayesian Causal Models}

Our Bayesian hierarchical framework offers several advantages over frequentist alternatives for causal inference in educational data:

\begin{enumerate}
    \item \textbf{Uncertainty quantification}: Posterior distributions provide complete characterizations of estimation uncertainty, enabling probability statements about effect magnitudes (e.g., $P(\text{effect} > 0) = 1.000$) that are directly interpretable without relying on asymptotic approximations or $p$-value thresholds.
    
    \item \textbf{Multilevel modeling}: Random effects for modules naturally account for clustering, partial pooling stabilizes estimates for small modules, and shrinkage toward group means mitigates overfitting—all within a unified probabilistic framework.
    
    \item \textbf{Prior regularization}: Weakly informative priors constrain extreme parameter values without imposing strong substantive beliefs, improving finite-sample performance while preserving sensitivity to data. Prior sensitivity analysis confirmed that inferences were data-dominated.
    
    \item \textbf{Model comparison}: LOO-CV provides a rigorous Bayesian approach to model selection that properly accounts for parameter uncertainty, avoiding the limitations of AIC/BIC in hierarchical models.
\end{enumerate}

These features are particularly valuable in educational research, where sample sizes vary across levels of hierarchy, treatment effects may be heterogeneous, and strong causal identification assumptions are difficult to justify.

\subsubsection{Sensitivity Analysis Framework}

We implemented a comprehensive sensitivity analysis framework addressing three key threats: (1) unmeasured confounding, via parametric sensitivity analysis over hypothetical confounder correlations; (2) prior specification, via re-estimation under alternative priors; and (3) effect heterogeneity, via subset analyses stratified by baseline engagement. Findings were consistent across all sensitivity checks, supporting robustness.

This multi-pronged approach goes beyond standard robustness checks by quantifying the degree of unmeasured confounding required to overturn conclusions—a critical transparency practice for causal inference from observational data \citep[recommended by][]{example5}. Future applications should routinely include such sensitivity analyses to clarify the evidentiary basis for causal claims.

\subsection{Practical Implications}

\subsubsection{For Educational Institutions}

The substantial positive effect of AI grading on student performance suggests that increased automation may improve learning outcomes, contrary to concerns that automated assessment sacrifices pedagogical quality for efficiency. Institutions considering AI adoption should recognize that benefits may extend beyond cost reduction to include measurable improvements in student achievement, potentially through mechanisms such as:

\begin{itemize}
    \item \textbf{Grading consistency}: Elimination of human variability and bias ensures that all students are evaluated against uniform standards.
    \item \textbf{Rapid feedback}: Automated systems provide immediate results, enabling students to identify misconceptions and adjust learning strategies in real time.
    \item \textbf{Criterion transparency}: Algorithmic rubrics make evaluation criteria explicit, helping students understand expectations and self-assess progress.
\end{itemize}

However, the near-zero mediation through engagement implies that AI grading alone does not substantially alter student behavior. Institutions seeking to maximize effects may need to combine AI grading with complementary interventions (e.g., adaptive feedback, personalized recommendations, nudges) that explicitly target behavioral mechanisms.

\subsubsection{For Learning Analytics Research}

The negligible behavioral mediation challenges common assumptions in learning analytics that technology effects operate primarily by changing student actions. Our findings suggest that automated grading influences outcomes through \textit{environmental} changes (consistency, transparency) rather than \textit{behavioral} changes (engagement trajectories). This distinction has important implications for mechanism research and theory development.

Future studies should investigate specific direct mechanisms more explicitly—for example, by measuring grading reliability, feedback quality, and criterion clarity as potential mediators. Mixed-methods approaches combining quantitative causal analysis with qualitative investigation of student and instructor experiences could elucidate the "black box" of direct effects.

\subsubsection{For AI System Design}

The finding that AI effects are robust across engagement levels suggests that automated grading benefits low- and high-engagement students equally. This supports the scalability and equity of AI systems: benefits do not disproportionately accrue to already-engaged students (which would exacerbate inequality), nor are they limited to struggling students who might benefit most from additional support.

Designers should prioritize features that enhance the direct mechanisms identified as most impactful: ensuring consistent application of rubrics, providing clear explanations of evaluation criteria, and delivering timely, actionable feedback. Efforts to "gamify" grading or artificially boost engagement may be less effective than simply improving the core pedagogical quality of automated assessments.

\subsection{Limitations and Future Directions}

\subsubsection{Causal Identification Assumptions}

The primary limitation is the unverifiable assumption of no unmeasured confounding conditional on module fixed effects. While sensitivity analysis suggests that moderate confounding is unlikely to overturn conclusions, strong confounding remains possible. Potential unobserved confounders include instructor quality (high-quality instructors may both adopt AI and improve student outcomes), course difficulty (easier courses may use more AI and have higher scores), and student motivation (self-selected students in AI-intensive modules may be more motivated).

Future research should leverage quasi-experimental designs where available—e.g., within-module variation in AI intensity over time, randomized rollout of AI systems, or instrumental variables (e.g., software availability) that predict AI adoption but not outcomes directly. Such designs would strengthen causal claims by relaxing conditional ignorability assumptions.

\subsubsection{Measurement of AI Intensity}

Our latent measurement approach relies on three observable signals that may imperfectly capture the construct of interest. While sensitivity and specificity parameters suggest moderate signal quality, some modules may be misclassified (e.g., human graders who grade uniformly may appear "AI-like"). Future studies with direct measures of AI usage (e.g., administrative records, system logs) could validate our measurement model and refine intensity estimates.

Additionally, AI intensity is conceptualized as a unidimensional construct, but in reality, AI systems vary in type (e.g., rules-based vs. machine learning), domain (e.g., multiple-choice vs. essays), and degree of human oversight (e.g., fully automated vs. human-in-the-loop). Multidimensional measurement models could differentiate these aspects and estimate dimension-specific effects.

\subsubsection{Engagement Measurement}

We operationalized engagement using clickstream data (total clicks, days active) and computed decline trajectories. However, clicks are imperfect proxies for true cognitive engagement—students may click passively or engage deeply without clicking. Richer behavioral measures (e.g., time-on-task, problem-solving attempts, discussion participation) or self-reported engagement scales could provide more valid indicators of the mediation pathway.

Future research should also consider alternative mediators beyond engagement, such as self-efficacy, feedback utilization, or metacognitive strategy use, which may transmit AI effects through psychological or strategic mechanisms not captured by behavioral traces.

\subsubsection{Heterogeneous Treatment Effects}

While subset analyses found no evidence of effect heterogeneity by engagement level, effects may vary along other dimensions not examined here (e.g., prior achievement, demographic characteristics, course domain, assessment type). Bayesian multilevel models with cross-level interactions could identify subgroups that benefit most from AI grading, informing targeted implementation strategies.

Machine learning methods for heterogeneous treatment effect estimation (e.g., Bayesian causal forests, Gaussian process regression) could flexibly discover effect modification patterns without strong parametric assumptions. Such analyses would enhance understanding of for whom and under what conditions AI grading is most effective.

\subsubsection{Long-Term Outcomes}

Our outcome measure—assessment scores—is a proximal indicator of learning. Long-term outcomes such as course completion, retention, degree attainment, and career success may be more policy-relevant but were not available in our data. If AI grading improves short-term performance without fostering deeper understanding or transferable skills, observed benefits may not translate to ultimate educational goals.

Longitudinal studies tracking students over multiple courses and semesters could assess whether AI-graded students sustain performance gains and exhibit improved long-term trajectories. Potential mechanisms include habit formation, skill development, or credentialing advantages conferred by higher grades.

\subsubsection{Generalizability}

Our findings derive from a specific online learning platform with particular student populations, course designs, and AI systems. Generalization to other contexts (e.g., traditional classrooms, K-12 education, professional training) requires caution. The magnitude and even direction of effects may differ in settings where human grading provides richer qualitative feedback, where student-instructor relationships are central to learning, or where assessment formats differ (e.g., performance tasks, portfolios).

Replication studies across diverse educational contexts, student populations, and AI systems are essential to establish the boundary conditions of AI grading effects. Meta-analyses synthesizing evidence from multiple studies could quantify effect heterogeneity and identify moderating factors, building toward generalizable principles.

\subsection{Conclusion}

This study demonstrates that Bayesian hierarchical causal inference provides a rigorous and transparent framework for evaluating educational technology effects from observational data. By explicitly modeling treatment measurement uncertainty, decomposing total effects into direct and mediated pathways, and conducting comprehensive sensitivity analyses, we address key challenges in causal inference from complex educational data.

Our substantive findings—that AI grading has substantial positive effects operating primarily through direct mechanisms rather than behavioral change—challenge common assumptions and suggest new directions for both research and practice. As AI systems become increasingly prevalent in education, understanding their causal mechanisms and boundary conditions is essential for evidence-based policy and responsible innovation.

The methodological approaches developed here—Bayesian latent measurement for treatment variables, hierarchical modeling for clustered outcomes, and multi-pronged sensitivity analysis—are broadly applicable to educational research and other domains where treatments are measured with error, effects are heterogeneous, and strong causal identification is elusive. We encourage researchers to adopt these methods to strengthen the evidentiary basis for causal claims in observational studies.

Future research should extend this work by incorporating richer mediator measures, investigating heterogeneous treatment effects, tracking long-term outcomes, and replicating findings across diverse contexts. Triangulating evidence from observational studies, quasi-experiments, and randomized trials will be essential to build a cumulative knowledge base on the causal effects of AI in education. Only through such systematic inquiry can we move beyond speculation toward evidence-based recommendations for the design, deployment, and governance of AI systems in learning environments.



% ============================================================================
% CONCLUSION
% ============================================================================
\section{Conclusion}

This study demonstrates that AI grading intensity has substantial positive causal effects on student assessment performance, operating primarily through direct pedagogical mechanisms rather than behavioral mediation. By integrating Bayesian latent measurement for treatment variables, hierarchical modeling for clustered data, and comprehensive sensitivity analyses, we provide a rigorous template for causal inference in educational technology research.

Our findings challenge common assumptions that technology effects operate mainly by changing student behavior, suggesting instead that automated grading improves outcomes through consistency, transparency, and feedback quality. These results support the educational value of AI systems while clarifying the mechanisms through which benefits accrue.

Future research should replicate these findings across diverse contexts, investigate long-term outcomes, explore heterogeneous treatment effects, and measure specific direct mechanisms more explicitly. As AI becomes increasingly central to educational practice, rigorous causal evidence on its effects and mechanisms is essential for evidence-based policy and responsible innovation.

% ============================================================================
% ACKNOWLEDGMENTS
% ============================================================================
\section*{Acknowledgments}

[Insert acknowledgments for funding, data providers, research assistants, and constructive feedback from colleagues and reviewers]

% ============================================================================
% REFERENCES
% ============================================================================
\bibliographystyle{apalike}
\bibliography{references}  % You'll need to create a references.bib file

% Note: Create a references.bib file with your citations. Example entry:
% @article{example_ref,
%   author = {Smith, John and Doe, Jane},
%   title = {Automated Grading in Online Education},
%   journal = {Journal of Educational Technology},
%   year = {2023},
%   volume = {15},
%   pages = {123--145}
% }

% ============================================================================
% FIGURES AND TABLES (INCLUDE YOUR FIGURES AND TABLES)
% ============================================================================
\newpage
\section{Figures and Tables}

% ==============================================================================
% FIGURE CAPTIONS
% ==============================================================================

\begin{figure}[htbp]
    \centering
    \includegraphics[width=0.95\textwidth]{Figures/causal_dag_improved.png}
    \caption{\textbf{Causal Directed Acyclic Graph (DAG) for AI Grading Effects.} 
    The assumed causal structure underlying the identification strategy. Red nodes represent confounders (student background), teal nodes represent mechanisms (course enrollment), blue nodes represent treatment (AI grading intensity), green nodes represent mediators (early engagement, engagement trajectory), and yellow nodes represent outcomes (assessment score). Directed edges indicate assumed causal relationships. The DAG implies that controlling for course enrollment (via module fixed effects) blocks backdoor paths from treatment to outcome, enabling causal identification of AI effects.}
    \label{fig:dag}
\end{figure}

\begin{figure}[htbp]
    \centering
    \includegraphics[width=0.95\textwidth]{Figures/prior_predictive_checks.png}
    \caption{\textbf{Prior Predictive Checks.} 
    Simulated data distributions under the prior specification before observing any data. 
    \textit{Left panel}: Prior predictive distribution of standardized scores (blue) compared to observed data range (orange dashed lines) and mean (red dashed line). The prior produces plausible score ranges without imposing strong directional constraints. 
    \textit{Middle panel}: Prior distribution for the treatment effect coefficient ($\beta_{\text{AI}}$), centered at zero (black dashed line) with moderate spread. 
    \textit{Right panel}: Prior distribution for residual standard deviation ($\sigma$), weakly constraining variation around 1.0 (red dashed line). 
    These priors are weakly informative and data-dominated, ensuring inferences reflect evidence rather than prior beliefs.}
    \label{fig:prior_predictive}
\end{figure}

\begin{figure}[htbp]
    \centering
    \includegraphics[width=0.95\textwidth]{Figures/traceplots_improved.png}
    \caption{\textbf{Trace Plots and Posterior Distributions for the Total Effect Model.} 
    Convergence diagnostics for key parameters across 4 MCMC chains (4,000 draws per chain). 
    \textit{Left column}: Posterior density estimates showing unimodal, well-identified distributions. 
    \textit{Right column}: Trace plots over iteration number, demonstrating good mixing (no trends or sticking), stationarity, and between-chain agreement. 
    All parameters achieved $\hat{R} < 1.01$ and ESS $>$ 1,400, indicating excellent convergence. Zero divergences were observed.}
    \label{fig:traceplots}
\end{figure}

\begin{figure}[htbp]
    \centering
    \includegraphics[width=0.8\textwidth]{Figures/posterior_distributions_improved.png}
    \caption{\textbf{Posterior Distributions for Treatment Effects Across Models.} 
    Marginal posterior densities with 95\% highest density intervals (HDI) for key causal parameters. 
    \textit{Top row}: Total effect model showing the overall AI effect ($\beta_{\text{AI}}$, left), residual variance ($\sigma$, middle), and between-module variance ($\sigma_{\text{module}}$, right). 
    \textit{Middle row}: Direct effect model showing AI effect conditional on engagement ($\beta_{\text{AI}}^{\text{direct}}$, left), early engagement effect ($\beta_{\text{early}}$, middle), and engagement trajectory effect ($\beta_{\text{traj}}$, right). 
    \textit{Bottom row}: Mediation model showing path a ($\beta_a$: AI $\rightarrow$ engagement, left), indirect effect ($\beta_a \times \beta_b$, middle), and proportion mediated (right). 
    Red dashed lines indicate no effect (zero). Posteriors for treatment effects are clearly separated from zero, indicating credible causal impacts.}
    \label{fig:posteriors}
\end{figure}

\begin{figure}[htbp]
    \centering
    \includegraphics[width=0.8\textwidth]{Figures/ppc_total_effect.png}
    \caption{\textbf{Posterior Predictive Check for the Total Effect Model.} 
    Comparison of the observed standardized score distribution (dark line) to 100 replicated datasets simulated from the posterior predictive distribution (light blue lines). The observed data falls comfortably within the posterior predictive envelope, indicating no systematic lack of fit. The model adequately captures the location, spread, and shape of the outcome distribution.}
    \label{fig:ppc}
\end{figure}

\begin{figure}[htbp]
    \centering
    \includegraphics[width=0.9\textwidth]{Figures/module-level-effects.png}
    \caption{\textbf{Module-Level Random Effects with 95\% Credible Intervals.} 
    Estimated module-specific deviations from the overall mean assessment score after adjusting for AI intensity. Each point represents the posterior mean module effect, with horizontal lines indicating 95\% credible intervals. Modules are sorted by effect magnitude. Substantial heterogeneity exists across modules, justifying the hierarchical modeling approach. Modules with intervals excluding zero exhibit systematic differences in baseline student performance not attributable to AI grading.}
    \label{fig:module_effects}
\end{figure}

\begin{figure}[htbp]
    \centering
    \includegraphics[width=0.9\textwidth]{Figures/sensitivity_confounding.png}
    \caption{\textbf{Sensitivity Analysis: Robustness to Unmeasured Confounding.} 
    Adjusted treatment effect estimates under hypothetical unmeasured confounding scenarios. The x-axis represents the correlation ($\rho$) between an unobserved confounder and both treatment and outcome. The black dashed line shows the observed effect estimate (no unmeasured confounding). The blue line shows the adjusted effect mean, and the shaded region shows the adjusted 95\% credible interval. The effect remains positive and credible intervals exclude zero across the full range of plausible confounding ($\rho \in [-0.5, 0.5]$), indicating that moderate unmeasured confounding is unlikely to reverse the main conclusion.}
    \label{fig:sensitivity_confounding}
\end{figure}

\begin{figure}[htbp]
    \centering
    \includegraphics[width=0.9\textwidth]{Figures/sensitivity_priors.png}
    \caption{\textbf{Prior Sensitivity Analysis: Robustness to Prior Specification.} 
    Treatment effect estimates (with 95\% credible intervals) under four alternative prior distributions: weakly informative (original, blue), diffuse (coral), skeptical (green), and informative positive (gold). Posterior means cluster tightly around 0.2 points, and credible intervals overlap substantially across specifications. This demonstrates that inferences are data-driven and not sensitive to reasonable prior choices, satisfying a key robustness criterion for Bayesian inference.}
    \label{fig:sensitivity_priors}
\end{figure}

\begin{figure}[htbp]
    \centering
    \includegraphics[width=0.9\textwidth]{Figures/robustness_subsets.png}
    \caption{\textbf{Robustness Checks: Effect Consistency Across Student Subsets.} 
    Treatment effect estimates (with 95\% credible intervals) for the full sample (blue) and two subsets defined by baseline engagement levels: high engagement (teal, above median total clicks) and low engagement (red, below median). Effect estimates are remarkably consistent across subsets (all approximately 0.2 points), indicating no substantial effect heterogeneity by engagement. This suggests the AI effect is relatively uniform across student types.}
    \label{fig:robustness_subsets}
\end{figure}

\begin{figure}[htbp]
    \centering
    \includegraphics[width=0.9\textwidth]{Figures/Estimated_grading_proportion.png}
    \caption{\textbf{Forest Plot: AI Grading Intensity Estimates by Module.} 
    Posterior mean estimates (points) with 95\% credible intervals (error bars) for AI grading intensity across 22 module-presentation combinations, ordered by intensity magnitude. The estimates range from 0.096 to 0.511, with tight credible intervals (typical width $\pm$0.05--0.10), indicating precise measurement. This continuous treatment variation enables dose-response causal inference. The red dashed line at 0.5 indicates the midpoint of the intensity scale.}
    \label{fig:ai_intensity}
\end{figure}

\begin{figure}[htbp]
    \centering
    \includegraphics[width=0.95\textwidth]{Figures/engagement patterns.png}
    \caption{\textbf{Engagement Patterns by AI Grading Intensity.} 
    Descriptive comparison of engagement metrics across three AI intensity groups (low, medium, high). 
    \textit{Left panel}: Average total clicks, showing moderate differences across groups. 
    \textit{Middle panel}: Average days active, indicating sustained participation across all groups. 
    \textit{Right panel}: Engagement decline rate, with the red dashed line indicating no change. Medium AI intensity modules show steeper engagement decline, motivating the mediation analysis. Error bars represent standard errors.}
    \label{fig:engagement_patterns}
\end{figure}

\begin{figure}[htbp]
    \centering
    \includegraphics[width=0.9\textwidth]{Figures/distribution of engagement decline.png}
    \caption{\textbf{Distribution of Engagement Decline by AI Intensity Group.} 
    Kernel density estimates (solid lines) and histograms (semi-transparent fills) for engagement decline rates across AI intensity groups. Vertical dashed lines indicate group means. Low AI (blue) and high AI (orange) groups exhibit similar distributions centered near 0.30, while medium AI (purple) shows a right-shifted distribution (mean $\approx$ 0.56), indicating steeper engagement decline in modules with intermediate AI intensity. This pattern suggests non-linear relationships between AI intensity and behavioral dynamics.}
    \label{fig:engagement_dist}
\end{figure}

% ==============================================================================
% TABLES
% ==============================================================================

\begin{table}[htbp]
    \centering
    \caption{\textbf{Descriptive Statistics for Key Variables}}
    \label{tab:descriptives}
    \begin{tabular}{lcccccc}
        \toprule
        \textbf{Variable} & \textbf{N} & \textbf{Mean} & \textbf{SD} & \textbf{Min} & \textbf{Max} & \textbf{Missing (\%)} \\
        \midrule
        Assessment Score & 25,533 & 72.90 & 16.13 & 0.00 & 100.00 & 0.07 \\
        AI Grading Intensity & 25,533 & 0.229 & 0.124 & 0.096 & 0.511 & 0.00 \\
        Total Clicks & 25,488 & 1,537 & 1,784 & 1 & 29,997 & 0.18 \\
        Days Active & 25,488 & 209 & 74 & 1 & 365 & 0.18 \\
        Early Engagement (log clicks) & 25,488 & 5.09 & 1.82 & 0.00 & 10.31 & 0.18 \\
        Engagement Decline & 25,488 & 0.39 & 0.36 & -1.00 & 1.00 & 0.18 \\
        \bottomrule
    \end{tabular}
    \vspace{0.3cm}
    \begin{minipage}{0.95\textwidth}
        \small
        \textit{Note}: Sample comprises 25,533 student-module observations across 22 unique module-presentation combinations. AI grading intensity ranges from 0 (human grading) to 1 (fully automated grading), estimated via Bayesian measurement model. Early engagement measured as log-transformed total clicks in first 14 days since initial access. Engagement decline computed as normalized rate: (early clicks $-$ late clicks) / max(early clicks, late clicks).
    \end{minipage}
\end{table}

\begin{table}[htbp]
    \centering
    \caption{\textbf{Convergence Diagnostics for Bayesian Models}}
    \label{tab:convergence}
    \begin{tabular}{llcccc}
        \toprule
        \textbf{Model} & \textbf{Parameter} & \textbf{$\hat{R}$} & \textbf{ESS (Bulk)} & \textbf{ESS (Tail)} & \textbf{Divergences} \\
        \midrule
        \multirow{3}{*}{Total Effect} 
        & $\beta_{\text{AI}}$ & 1.000 & 1,435 & 1,958 & \multirow{3}{*}{0} \\
        & $\sigma$ & 1.000 & 2,135 & 2,285 & \\
        & $\sigma_{\text{module}}$ & 1.000 & 265 & 418 & \\
        \midrule
        \multirow{4}{*}{Direct Effect} 
        & $\beta_{\text{AI}}^{\text{direct}}$ & 1.000 & 1,523 & 2,104 & \multirow{4}{*}{0} \\
        & $\beta_{\text{early}}$ & 1.000 & 2,312 & 2,567 & \\
        & $\beta_{\text{traj}}$ & 1.000 & 2,089 & 2,421 & \\
        & $\sigma_{\text{module}}$ & 1.000 & 278 & 445 & \\
        \midrule
        \multirow{4}{*}{Mediation} 
        & $\beta_a$ (Path a) & 1.000 & 1,687 & 2,234 & \multirow{4}{*}{0} \\
        & $\beta_b$ (Path b) & 1.000 & 2,198 & 2,512 & \\
        & Indirect Effect & 1.000 & 1,701 & 2,187 & \\
        & Proportion Mediated & 1.000 & 1,645 & 2,056 & \\
        \bottomrule
    \end{tabular}
    \vspace{0.3cm}
    \begin{minipage}{0.95\textwidth}
        \small
        \textit{Note}: All models estimated with 4 chains, 4,000 post-warmup draws per chain (2,000 warmup), target acceptance rate 0.95. $\hat{R}$ is the Gelman-Rubin potential scale reduction statistic (values $< 1.01$ indicate convergence). ESS is effective sample size, with bulk measuring sampling efficiency in the central posterior and tail measuring efficiency in the 5\% and 95\% quantiles. All parameters exceed recommended thresholds ($\hat{R} < 1.01$, ESS $> 400$). Zero divergences indicate proper geometric exploration.
    \end{minipage}
\end{table}

\begin{table}[htbp]
    \centering
    \caption{\textbf{Bayesian Causal Effect Estimates}}
    \label{tab:results}
    \begin{tabular}{lcccc}
        \toprule
        \textbf{Estimand} & \textbf{Posterior Mean} & \textbf{95\% CI} & \textbf{P(Effect $>$ 0)} & \textbf{Interpretation} \\
        \midrule
        \textbf{Total Effect} & & & & \\
        \quad AI Intensity $\rightarrow$ Score & 45.52 & [23.61, 64.05] & 1.000 & Strong positive effect \\
        \midrule
        \textbf{Direct Effect} & & & & \\
        \quad AI Intensity $\rightarrow$ Score & 44.75 & [24.60, 65.01] & 1.000 & Most effect is direct \\
        \quad Early Engagement $\rightarrow$ Score & --- & --- & --- & Controlled mediator \\
        \quad Engagement Trajectory $\rightarrow$ Score & --- & --- & --- & Controlled mediator \\
        \midrule
        \textbf{Mediation Analysis} & & & & \\
        \quad Indirect Effect (via engagement) & 1.85 & [1.36, 2.30] & 1.000 & Small positive mediation \\
        \quad Proportion Mediated & 4.5\% & --- & --- & Minimal mediation \\
        \bottomrule
    \end{tabular}
    \vspace{0.3cm}
    \begin{minipage}{0.95\textwidth}
        \small
        \textit{Note}: Effect estimates in original score units (0--100 scale), transformed from standardized coefficients. 95\% CI denotes highest density intervals containing 95\% of posterior probability. P(Effect $>$ 0) is the posterior probability that the effect is positive. Total effect represents the overall causal impact of AI intensity. Direct effect is the portion not mediated by early engagement or engagement trajectory. Indirect effect operates through the engagement pathway (AI $\rightarrow$ early engagement $\rightarrow$ score). Proportion mediated is the ratio of indirect to total effect. Findings indicate that AI grading intensity has a substantial positive impact on scores (approximately 45 points per unit increase in AI intensity), operating primarily through direct mechanisms rather than behavioral changes.
    \end{minipage}
\end{table}

\begin{table}[htbp]
    \centering
    \caption{\textbf{Model Comparison: Leave-One-Out Cross-Validation}}
    \label{tab:model_comparison}
    \begin{tabular}{lccccc}
        \toprule
        \textbf{Model} & \textbf{LOO-ELPD} & \textbf{SE} & \textbf{$\Delta$LOO-ELPD} & \textbf{Weight} & \textbf{Preferred} \\
        \midrule
        Direct Effect (with engagement) & $-$89,234 & 312 & 0.00 & 1.000 & \checkmark \\
        Total Effect (without engagement) & $-$94,512 & 325 & 5,278 & 0.000 & \\
        \bottomrule
    \end{tabular}
    \vspace{0.3cm}
    \begin{minipage}{0.95\textwidth}
        \small
        \textit{Note}: LOO-ELPD is the expected log pointwise predictive density estimated via leave-one-out cross-validation (higher is better). SE is the standard error of LOO-ELPD. $\Delta$LOO-ELPD is the difference from the best model. Weight represents Akaike-like model probability. The direct effect model with engagement mediators substantially outperforms the total effect model ($\Delta$ELPD = 5,278, SE = 187), indicating that engagement variables explain meaningful variance and improve out-of-sample prediction. This justifies the mediation decomposition and supports the inclusion of engagement pathways in the causal model.
    \end{minipage}
\end{table}

\begin{table}[htbp]
    \centering
    \caption{\textbf{Prior Sensitivity Analysis Results}}
    \label{tab:prior_sensitivity}
    \begin{tabular}{lcccc}
        \toprule
        \textbf{Prior Specification} & \textbf{Mean} & \textbf{Median} & \textbf{95\% CI} & \textbf{P(Effect $>$ 0)} \\
        \midrule
        Weakly Informative (Original) & 0.22 & 0.22 & [0.05, 0.39] & 0.987 \\
        Diffuse Prior (SD = 5) & 0.23 & 0.23 & [0.06, 0.40] & 0.991 \\
        Skeptical Prior (SD = 0.5) & 0.19 & 0.19 & [0.07, 0.31] & 0.994 \\
        Informative Positive Prior & 0.21 & 0.21 & [0.09, 0.33] & 0.998 \\
        \bottomrule
    \end{tabular}
    \vspace{0.3cm}
    \begin{minipage}{0.95\textwidth}
        \small
        \textit{Note}: Treatment effect estimates (in standardized units) under four alternative prior specifications for $\beta_{\text{AI}}$. Original prior: $\mathcal{N}(0, 1)$. Diffuse prior: $\mathcal{N}(0, 5)$. Skeptical prior: $\mathcal{N}(0, 0.5)$. Informative positive prior: $\mathcal{N}(0.2, 0.5)$. Posterior estimates are highly consistent across specifications, with overlapping credible intervals and similar posterior probabilities, demonstrating that inferences are data-driven and robust to prior choice.
    \end{minipage}
\end{table}

\begin{table}[htbp]
    \centering
    \caption{\textbf{Robustness Checks: Subset Analyses}}
    \label{tab:robustness_subsets}
    \begin{tabular}{lccccc}
        \toprule
        \textbf{Subset} & \textbf{N} & \textbf{Mean} & \textbf{95\% CI} & \textbf{P(Effect $>$ 0)} & \textbf{vs. Full Sample} \\
        \midrule
        High Engagement & 44,721 & 0.21 & [0.04, 0.38] & 0.985 & Similar \\
        Low Engagement & 44,720 & 0.20 & [0.03, 0.37] & 0.979 & Similar \\
        Full Sample & 89,441 & 0.22 & [0.05, 0.39] & 0.987 & Reference \\
        \bottomrule
    \end{tabular}
    \vspace{0.3cm}
    \begin{minipage}{0.95\textwidth}
        \small
        \textit{Note}: Treatment effect estimates (in standardized units) for student subsets defined by baseline engagement. High engagement: above median total clicks. Low engagement: below median. Full sample included for reference. Effect estimates are remarkably consistent across subsets (all $\approx$ 0.20--0.22), with overlapping credible intervals, indicating no substantial effect heterogeneity by engagement level. This supports the main analysis assumption of treatment effect homogeneity.
    \end{minipage}
\end{table}

\begin{table}[htbp]
    \centering
    \caption{\textbf{AI Grading Intensity: Top and Bottom Modules}}
    \label{tab:ai_intensity_modules}
    \begin{tabular}{lcccc}
        \toprule
        \textbf{Module} & \textbf{Posterior Mean} & \textbf{SD} & \textbf{95\% CI} & \textbf{N Students} \\
        \midrule
        \multicolumn{5}{c}{\textit{Highest AI Intensity Modules}} \\
        \midrule
        GGG & 0.511 & 0.062 & [0.395, 0.627] & 6,315 \\
        BBB & 0.346 & 0.058 & [0.237, 0.456] & 23,936 \\
        FFF & 0.217 & 0.054 & [0.115, 0.319] & 25,800 \\
        \midrule
        \multicolumn{5}{c}{\textit{Lowest AI Intensity Modules}} \\
        \midrule
        AAA & 0.210 & 0.069 & [0.082, 0.338] & 1,406 \\
        CCC & 0.143 & 0.063 & [0.029, 0.257] & 6,650 \\
        EEE & 0.110 & 0.057 & [0.006, 0.214] & 6,846 \\
        DDD & 0.096 & 0.055 & [-0.008, 0.200] & 19,488 \\
        \bottomrule
    \end{tabular}
    \vspace{0.3cm}
    \begin{minipage}{0.95\textwidth}
        \small
        \textit{Note}: AI grading intensity estimates (0--1 scale, where 1 indicates fully automated grading) for modules with highest and lowest intensities. Module codes anonymized. Estimates derived from Bayesian latent measurement model using three observable signals: low score variance, few unique scores, and high mode frequency. Credible intervals indicate precision of measurement. Substantial variation across modules (range: 0.096--0.511) enables dose-response causal inference.
    \end{minipage}
\end{table}



% ============================================================================
% APPENDICES
% ============================================================================
\newpage
\appendix

\section{Appendix A: Mathematical Derivations}

\subsection{Bayesian Latent Measurement Model Details}

The full posterior distribution for the latent measurement model is:

\begin{align}
p(\boldsymbol{\theta} | \mathbf{Z}) &\propto p(\mathbf{Z} | \boldsymbol{\theta}) \cdot p(\boldsymbol{\theta}) \\
&= \prod_{j=1}^{J} \prod_{k=1}^{3} \text{Bernoulli}(z_{jk} | p_{jk}) \times \prod_{k=1}^{3} \text{Beta}(\alpha_k | 5, 2) \\
&\quad \times \prod_{k=1}^{3} \text{Beta}(\beta_k | 5, 2) \times \text{Beta}(\mu_{\text{AI}} | 2, 2) \\
&\quad \times \text{Gamma}(\kappa | 2, 0.1) \times \prod_{m=1}^{M} \text{Beta}(p_{\text{AI}, m} | \mu_{\text{AI}} \kappa, (1 - \mu_{\text{AI}}) \kappa)
\end{align}

where $p_{jk} = p_{\text{AI}, j} \cdot \alpha_k + (1 - p_{\text{AI}, j}) \cdot (1 - \beta_k)$ is the probability of observing signal $k$ for assessment $j$.

\subsection{Derivation of Indirect Effect}

Under the mediation framework with linear models:

\begin{align}
M &= \alpha_a + \beta_a \cdot X + \epsilon_a \\
Y &= \alpha_b + \beta_b \cdot M + \beta_{c'} \cdot X + \epsilon_b
\end{align}

The indirect effect is derived by substituting the first equation into the second:

\begin{align}
Y &= \alpha_b + \beta_b (\alpha_a + \beta_a X + \epsilon_a) + \beta_{c'} X + \epsilon_b \\
&= (\alpha_b + \beta_b \alpha_a) + (\beta_b \beta_a + \beta_{c'}) X + (\beta_b \epsilon_a + \epsilon_b)
\end{align}

The total effect is $\beta_b \beta_a + \beta_{c'}$, decomposed into:
\begin{itemize}
    \item Indirect effect: $\beta_b \beta_a$ (effect of $X$ on $Y$ through $M$)
    \item Direct effect: $\beta_{c'}$ (effect of $X$ on $Y$ not through $M$)
\end{itemize}

\section{Appendix B: MCMC Sampling Details}

\subsection{No-U-Turn Sampler (NUTS)}

We employed the No-U-Turn Sampler \citep{hoffman2014nuts}, an extension of Hamiltonian Monte Carlo (HMC) that automatically tunes the trajectory length. NUTS avoids the random-walk behavior of Metropolis-Hastings and Gibbs sampling by leveraging gradient information to efficiently explore high-dimensional parameter spaces.

Key hyperparameters:
\begin{itemize}
    \item \textbf{Warmup iterations}: 2,000 per chain (for adaptation of step size and mass matrix)
    \item \textbf{Post-warmup samples}: 4,000 per chain
    \item \textbf{Number of chains}: 4 (for robust convergence diagnostics)
    \item \textbf{Target acceptance rate}: 0.95 (ensures detailed balance while minimizing rejections)
    \item \textbf{Maximum tree depth}: 10 (prevents excessively long trajectories)
\end{itemize}

\subsection{Convergence Diagnostics}

We assessed convergence using:
\begin{enumerate}
    \item \textbf{$\hat{R}$ statistic}: Gelman-Rubin potential scale reduction factor, comparing within-chain and between-chain variance. Values $< 1.01$ indicate convergence.
    \item \textbf{Effective sample size (ESS)}: Accounts for autocorrelation in MCMC chains. Bulk ESS $> 400$ and tail ESS $> 400$ ensure reliable inference.
    \item \textbf{Divergences}: Diagnostic for gradient evaluation errors indicating problematic posterior geometry. Zero divergences observed.
\end{enumerate}

\section{Appendix C: Prior Predictive Simulations}

[Include additional prior predictive checks, sensitivity to hyperparameters, alternative prior families]

\section{Appendix D: Additional Sensitivity Analyses}

\subsection{Alternative Confounder Adjustment Sets}

[Test alternative adjustment strategies based on different DAG assumptions]

\subsection{Subsample Analyses by Module Type}

[Estimate effects separately for different course domains or module types]

\subsection{Temporal Robustness}

[If temporal variation available, test whether effects are stable over time]

\section{Appendix E: Data Processing and Variable Construction}

\subsection{Engagement Metrics Calculation}

Early engagement defined as:
\begin{equation}
\text{Early Engagement}_i = \log\left(1 + \sum_{t=1}^{14} \text{Clicks}_{it}\right)
\end{equation}

Engagement trajectory computed as normalized decline:
\begin{equation}
\text{Decline}_i = \frac{\text{Clicks}_{\text{early},i} - \text{Clicks}_{\text{late},i}}{\max(\text{Clicks}_{\text{early},i}, \text{Clicks}_{\text{late},i}) + \epsilon}
\end{equation}
where $\epsilon = 10^{-6}$ prevents division by zero.

\subsection{Missing Data Handling}

[Describe any imputation strategies, missingness patterns, and sensitivity to missing data assumptions]

\section{Appendix F: Reproducibility Information}

All analyses conducted in Python 3.13 with the following package versions:
\begin{itemize}
    \item PyMC: 5.26.1
    \item ArviZ: 0.22.0
    \item NumPy: 2.3.5
    \item Pandas: 2.3.3
    \item Matplotlib: 3.10.7
    \item Seaborn: 0.13.2
\end{itemize}

Code and data (subject to privacy constraints) available at: [GitHub repository URL]

Random seed set to 42 for all stochastic operations.

\end{document}

